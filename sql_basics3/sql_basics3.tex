\documentclass{article}
\begin{document}
    relationships:
    \begin{itemize}
        \item 1to1
        \item 1to many
        \item many to many 
    \end{itemize}

    keys:
    \begin{itemize}
        \item Primary - unique key that identifies row in table
        \item Candidate - set of attributes that uniquely id the row, shouldnt have any redundant data
        \item super - same as Candidate but can have redundant data
        \item alternate - alternate to Primary
        \item simple key - one column
        \item composite - multi column
    \end{itemize}

    NATURAL Primary key - primary key that has some info about row
    surrogate pk - is fully artificial.

    surrogate is preffered because it is smaller,
    doesnt require cascading changes and can be generated if some part of row is null.

    Foreign key is column that refers to PK in another table.
    Recursive FK refers to PK in same table.

    constraints are used to specify rules for data in table and its columns.
    \begin{itemize}
        \item NOT NULL
        \item UNIQUE
        \item PRIMARY KEY
        \item FOREIGN KEY
        \item CHECK - specific condition
        \item DEFAULT - specifies non null value if it is not provided
        \item INDEX - used to create and retrieve data from db very fast.
    \end{itemize}

    CREATE TABLE <t name> ( \\
        <col name> <type> <constraint: NOT NULL>, \\
        <col> <type> DEFAULT <def val>,
        <col> <type> IDENTITY(<start>, <incr>) ...,
        ...\\
        CONSTRAINT <constraint name> UNIQUE(<cols>), \\
        PRIMARY KEY(<cols>), \\
        FOREIGN KEY(<cols>) REFERENCES <table name>(<cols>), \\
        CHECK (<condition>)
    );

    ALTER TABLE \\
    <ADD|DROP> <constraint>;

    CONSTRAINT <constraint name> only needed if you need to add name for constraint.
    only one identity column per table

    Window funcitons operate on a set of rows and return a sipnle value for each row in the underlying query
    (e.g. sets same value for column for n rows).

    SELECT \\
    <value> OVER(<partition by order by>) AS <alias> \\
    FROM <t>;

    PARTITION BY - divides whole query in groups of rows with same value in column(s)

    OVER(partition by order by ROWS)

    ROWS: \\
    SELECT Date \\
         , Medium​ \\
         , Conversions​ \\
         , SUM(Conversions) OVER( \\
            PARTITION BY Date \\
            ORDER BY Conversions \\
            ROWS BETWEEN CURRENT ROW AND 1 FOLLOWING -- 'Sum' will be SUM() of current and next row \\
         ) AS 'Sum' \\
    FROM Orders

    window functions:
    \begin{itemize}
        \item Aggregate - MAX, MIN, SUM, AVG, COUNT
        \item Ranking - RANK, DENSE_RANK, ROW_NUMBER, NTILE
        \item LAG, LEAD, FIRST_VALUE, LAST_VALUE, CUME_DIST.
    \end{itemize}
    \begin{itemize}
        \item ROW_NUMBER - just the row number(from 1 ascending)
        \item RANK - kinda like row number but if row has same value as previous it will have same RANK,
            but new value always has ROW_NUMBER as its RANK
        \item DENSE_RANK - same as RANK but new RANK is not ROW_NUMBER but previous + 1
        \item NTILE - splits into n row groups
        \item FIRST_VALUE - returns first value of specified query(column)
        \item LAST_VALUE
        \item LAG - kinda like python's pairwise but with default value(LAG(1, 2, 3) -> (0, 1), (1, 2), (2, 3))
        \item LEAD - same as LAG but the other way around (LEAD(1, 2, 3) -> (1, 2), (2, 3), (3, 0))
    \end{itemize}

    GETDATE - current time
    DATEPART - returns part of date/time (year/month/second etc)
    DATEADD - shift timestamp
    DATEDIFF - difference between two dates
    CONVERT - contverts expression to datatype according to style

    String functions:
    \begin{itemize}
        \item CONCAT
        \item CHAR - returns char based on code
        \item LEFT - first n characters
        \item LEN - length of str
        \item REPLACE
        \item REVERSE
        \item TRIM - removes whitespace and other specified characters from start and end of string
        \item UPPER - upper case
        \item STUFF - removes n characters at ith index and inserts new substing in their place
    \end{itemize}
\end{document}
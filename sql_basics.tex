\documentclass{article}
\usepackage{hyperref}

\begin{document}
    NoSQL db examples:
    \begin{itemize}
        \item Document model:
            These NoSQL databases replace the familiar rows and columns structure with a document storage model. \\
            Each document is structured, frequently using the JavaScript Object Notation (JSON) model.
            The document data model is associated with object - oriented programming 
            where each document is an object (use cases: catalog).
        \item Graph model: This class of databases uses structures like data modes,
            edges and properties, making it easier to model relationships
            between entities in an application (use cases: recommendation engines).
        \item Key-value model: In this NoSQL database model, a key is required to retrieve and update data.
            The key-value data model is very simple and therefore scales well.
            However, this simplicity and scalability come at the cost of query complexity (use cases: shopping cart).
        \item Wide-column model: Wide-column stores use the typical tables, columns,
            and rows, but unlike relational databases (RDBs), columnal formatting 
            and names can vary from row to row inside the same table.
            And each column is stored separately on disk.
    \end{itemize}

    CREATE DATABASE <database name>;

    SELECT * FROM <database name>; \\
    * means select all columns

    DROP DATABASE <database name>; \\
    deletes database

    BACKUP DATABASE <database name> \\
    TO DISK = '<path>';

    BACKUP DATABASE <database name> \\
    TO DISK = '<path>' \\
    WITH DIFFERENTIAL;

    Schema is kinda namespace inside database
    \begin{verbatim}
        CREATE SCHEMA <schema name>
        AUTHORIZATION <owner name>;

        AUTHORIZATION part is not necessary

        ALTER SCHEMA <schema name>
        TRANSFER <entity type>.<securable name>;

        <entity type>. is not necessary

        DROP SCHEMA <schema name>;

        CREATE TABLE <table_name>(
            <column name> <column type>,
            <column name> <column type>,
            <column name> <column type>,
            ...
        )

        DROP TABLE <table name>;

        TRUNCATE TABLE <table name>;

        removes all rows but not table

        ALTER TABLE <table name>
        ADD <column name> <type>;

        ALTER TABLE <table name>
        DROP COLUMN <column name>;

        ALTER TABLE <table name>
        ALTER COLUMN <column name> <type>;
    \end{verbatim}

    \begin{verbatim}
        INSERT INTO <table name>
            (<column name 1>,  <column name 2>, ...)
        VALUES                  
            (<value for col1>, <value for col2>, ...);

        INSERT INTO <table name>
        VALUES (<value for col1>, ...);

        this one is used if you provide values for all columns

        UPDATE <table name>
        SET <column name> = <column value>, ...
        WHERE <condition>;

        if where is ommited update is for whole table

        DELETE FROM <table name>
        WHERE <condition>;

        if where is ommited its same as truncate

        SELECT <col name 1>, <col name 2>, ...
        FROM <table name>;

        SELECT DISTINCT <col name 1>, <col name 2>, ...
        FROM <table name>;

        only returns distinct values
    \end{verbatim}

    Order of operations:
    \begin{itemize}
        \item from
        \item on
        \item join
        \item where
        \item group by
        \item having
        \item select
        \item distinct
        \item order by
        \item top
    \end{itemize}

    \begin{verbatim}
        SELECT <* or coma separated column names>
        INTO <new table name>
        FROM <old table name>
        WHERE <condition>;

        copies data from one table to new one
        where can be ommited

        INSERT INTO <t2>
        SELECT <* or names> FROM <t1>
        WHERE condition;

        copies data from one table to existing one

        WHERE <condition>;
        where clause is used to filter query
    \end{verbatim}

    where clause operators:
    \begin{itemize}
        \item = eq, e.g. SELECT <col1>, ... FROM <table> WHERE <col2> = <something>;
        \item > gt, example is same as for eq, and actually for other comparison operators
        \item < lt
        \item >= ge
        \item <= le
        \item <> ne
        \item BETWEEN inclusive range, e.g. WHERE <col> BETWEEN <x0> AND <x1>;
        \item LIKE pattern match, e.g. WHERE <col> LIKE \begin{verbatim}
            'a%'; % - matches anything,
            _ - matches single char
        \end{verbatim}
        \item IN multiple allowed values, e.g. WHERE <col> IN (<val1>, <val2>, ...);
    \end{itemize}

    AND, OR, NOT, can be used in WHERE clause

    Logical Operators Precedence:
    \begin{itemize}
        \item Parentheses
        \item Mul div
        \item Sub Add
        \item NOT
        \item AND
        \item OR
    \end{itemize}

    ORDER BY is used to sort output
    SELECT <cols>
    FROM <table>
    ORDER BY <col1> <ASC|DESC>,...;

    <ASC|DESC> can be skipped, default order is ASC;

    SELECT TOP <number> PERCENT <cols>
    FROM ...;

    only <number>\% of rows will be returned, if PERCENT is ommited than <number> of rows will be returned

    SELECT <cols>
    FROM <table>
    ORDER BY <cols ASC|DESC>
    OFFSET <n> <ROW|ROWS>
    FETCH <FIRST|NEXT> <m> <ROW|ROWS> ONLY;

    OFFSET skips n rows before starting to return
    FETCH returns m rows after offset

    FIRST and NEXT are synonyms

    FETCH is optiona
    OFFSET must be used with order by.

    comparison for null:
    IS NULL
    IS NOT NULL

    SELECT <col name> AS <col alias>
    FROM <table name> AS <table alias>;

    SELECT <col1> + ',' + <col2> + ' ' + <col3> AS <col4>
    FROM <table>;

    col1, 2 and 3 combined into one with different separators

    SELECT <t1>.<c1>, <t2>.<c1>, ...
    FROM <table1> as <t1>, <table2> as <t2>;
    -- this is a comment
    /*
    this is also a comment
    */

\end{document}
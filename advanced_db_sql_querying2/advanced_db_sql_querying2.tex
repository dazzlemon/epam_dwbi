\documentclass{article}
\usepackage[paperwidth=841pt,paperheight=595pt,top=28pt,right=28pt,bottom=28pt,left=28pt, includefoot, includehead]{geometry}
\usepackage{xcolor,listings}
\usepackage{textcomp}
\usepackage{color}
\usepackage{hyperref}

\definecolor{codegreen}{rgb}{0,0.6,0}
\definecolor{codegray}{rgb}{0.5,0.5,0.5}
\definecolor{codepurple}{HTML}{C42043}
\definecolor{backcolour}{HTML}{F2F2F2}
\definecolor{bookColor}{cmyk}{0,0,0,0.90}  
\color{bookColor}

\lstset{upquote=true}

\lstdefinestyle{mystyle}{
    backgroundcolor=\color{backcolour},   
    commentstyle=\color{codegreen},
    keywordstyle=\color{codepurple},
    numberstyle=\numberstyle,
    stringstyle=\color{codepurple},
    basicstyle=\footnotesize\ttfamily,
    breakatwhitespace=false,
    breaklines=true,
    captionpos=b,
    keepspaces=true,
    numbers=left,
    numbersep=10pt,
    showspaces=false,
    showstringspaces=false,
    showtabs=false,
}
\lstset{style=mystyle}

\newcommand\numberstyle[1]{%
    \footnotesize
    \color{codegray}%
    \ttfamily
    \ifnum#1<10 0\fi#1 |%
}

\begin{document}
    \begin{section}{user defined functions(UDF)}
        types:
        \begin{itemize}
            \item Scalar - returns a single value
            \item Table valued - returns a table
            \item System - built ins
        \end{itemize}

        \begin{lstlisting}[ language=SQL
                          , deletekeywords={IDENTITY}
                          , deletekeywords={[2]INT}
                          , morekeywords={clustered}
                          , framesep=8pt
                          , xleftmargin=40pt
                          , framexleftmargin=40pt
                          , frame=tb
                          , framerule=0pt 
                          ]
CREATE FUNCTION <fname> (<params>)
RETURNS <ftype>
AS BEGIN
    DECLARE <result> <ftype>
    SELECT  <result> = <statement>
    RETURN  <result>
END
        \end{lstlisting}
        udf agrregates: \\
        \url{https://docs.microsoft.com/en-us/sql/relational-databases/user-defined-functions/create-user-defined-aggregates?view=sql-server-ver15}
    
        \begin{lstlisting}[ language=SQL
                          , deletekeywords={IDENTITY}
                          , deletekeywords={[2]INT}
                          , morekeywords={clustered}
                          , framesep=8pt
                          , xleftmargin=40pt
                          , framexleftmargin=40pt
                          , frame=tb
                          , framerule=0pt 
                          ]
ALTER FUNCTION <f>
...

IF OBJECT_ID (<fname>, N'IF') IS NOT NULL
    DROP FUNCTION <fname>
        \end{lstlisting}

    u cant rename udfs \\
    udf disadvantages:
    \begin{itemize}
        \item CANNOT BE USED TO PERFORM ACTIONS THAT MODIFY THE DATABASE STATE
        \item CAN NOT RETURN MULTIPLE RESULT SETS
        \item UDFs dont support error handling
        \item CANNOT CALL stored procedure
        \item cannot make use of dynamic sql or temp tables
        \item FOR XML is not allowed
        \item cannot call another $>32$ deep
    \end{itemize}
    \end{section}
    \begin{section}{transactions and error handling}

    Transaction is a single unit of work \\
    if its successful the result is commited and its permanent \\
    otherwise it can be canceled or rolled back

    ACID:
    \begin{itemize}
        \item Atomicity
        \item Isolation
        \item Consistency
        \item Durability
    \end{itemize}

    \begin{lstlisting}[ language=SQL
                      , deletekeywords={IDENTITY}
                      , deletekeywords={[2]INT}
                      , morekeywords={clustered}
                      , framesep=8pt
                      , xleftmargin=40pt
                      , framexleftmargin=40pt
                      , frame=tb
                      , framerule=0pt 
                      ]
BEGIN <TRAN | TRANSACTION> <tname>
WITH MARK [ '<description>' ]
...
-- tname and description are optional
-- tname can be variable

COMMIT <TRAN | TRANSACTION> <tname>
-- tname is optional

ROLLBACK <TRAN | TRANSACTION> <tname | savepoint>

SAVE <TRAN | TRANSACTION> <savepoint>

SET TRANSACTION ISOLATION LEVEL <level>
    \end{lstlisting}

        Isolation levels:
        \begin{itemize}
            \item READ UNCOMMITTED
            \item READ COMMITTED - default
            \item REPEATABLE READ
            \item SNAPSHOT
            \item SERIALIZABLE
        \end{itemize}

        
        \begin{lstlisting}[ language=SQL
                          , deletekeywords={IDENTITY}
                          , deletekeywords={[2]INT}
                          , morekeywords={clustered}
                          , framesep=8pt
                          , xleftmargin=40pt
                          , framexleftmargin=40pt
                          , frame=tb
                          , framerule=0pt 
                          ]
@@error = 0 -- success
@@error = 1 -- fail

RAISERROR(<msg>, <severity>, <state>)

BEGIN TRY
    <statement>
END TRY
BEGIN CATCH
    <statement>
END CATCH
        \end{lstlisting}

        Data available in catch:
        \begin{itemize}
            \item ERROR\_NUMBER – returns the internal number of the error
            \item ERROR\_STATE – returns the information about the source
            \item ERROR\_SEVERITY – returns the information about anything from informational errors to errors 
                user of DBA can fix, etc.
            \item ERROR\_LINE – returns the line number at which an error happened on
            \item ERROR\_PROCEDURE – returns the name of the stored procedure or function
            \item ERROR\_MESSAGE – returns the most essential information and that is the message text of the error
        
        \end{itemize}
        

    \end{section}
    \begin{section}{dynamic queries}
    
    \end{section}
    \begin{section}{filestream and free text search}
    
    \end{section}
\end{document}
\documentclass{article}
\usepackage[paperwidth=841pt,paperheight=595pt,top=28pt,right=28pt,bottom=28pt,left=28pt, includefoot, includehead]{geometry}
\usepackage{xcolor,listings}
\usepackage{textcomp}
\usepackage{color}
\usepackage{hyperref}

\definecolor{codegreen}{rgb}{0,0.6,0}
\definecolor{codegray}{rgb}{0.5,0.5,0.5}
\definecolor{codepurple}{HTML}{C42043}
\definecolor{backcolour}{HTML}{F2F2F2}
\definecolor{bookColor}{cmyk}{0,0,0,0.90}  
\color{bookColor}

\lstset{upquote=true}

\lstdefinestyle{mystyle}{
    backgroundcolor=\color{backcolour},   
    commentstyle=\color{codegreen},
    keywordstyle=\color{codepurple},
    numberstyle=\numberstyle,
    stringstyle=\color{codepurple},
    basicstyle=\footnotesize\ttfamily,
    breakatwhitespace=false,
    breaklines=true,
    captionpos=b,
    keepspaces=true,
    numbers=left,
    numbersep=10pt,
    showspaces=false,
    showstringspaces=false,
    showtabs=false,
}
\lstset{style=mystyle}

\newcommand\numberstyle[1]{%
    \footnotesize
    \color{codegray}%
    \ttfamily
    \ifnum#1<10 0\fi#1 |%
}

\begin{document}
    \begin{section}{Stored Procedures}
        stored procedure(SP) - sql code that can be run multiple times with different params \\
        types of SPs:
        \begin{itemize}
            \item user defined
            \item temporary - tempdb
            \item system
        \end{itemize}

        \begin{lstlisting}[ language=SQL
                          , deletekeywords={IDENTITY}
                          , deletekeywords={[2]INT}
                          , morekeywords={clustered}
                          , framesep=8pt
                          , xleftmargin=40pt
                          , framexleftmargin=40pt
                          , frame=tb
                          , framerule=0pt 
                          ]
CREATE PROCEDURE <pname> <@parname partype, ...>
AS <sql statement>
<RETURN ...> -- optional

EXEC <pname> <@parname = parval, ...>
EXEC @val = <pname> <params>

SET NOCOUNT ON -- wont show how many rows affected

@param type OUTPUT -- in create 
@param OUTPUT -- in EXEC, @param should be declared with same type

SET ANSI_NULLS ON -- NULL comparisons always return NULL(as it is intended)
SET QUOTED_IDENTIFIER ON -- allows to use quotes same way as brackets, otherwise thay are same as apostrophes

ALTER PROCEDURE <pname>
AS <statement>

DROP PROCEDURE <pname>

IF OBJECT_ID ('<pname>', 'P') IS NOT NULL
    DROP PROCEDURE <pname>

EXEC sp_rename '<old pname>', '<new pname>'

sp_help -- info about db objs

-- if # before <pname> it is a local temporary proc
-- if ## it is global temp proc
        \end{lstlisting}
    \end{section}
    \begin{section}{computed columns}
        Computed column is column that aggregates existing columns, so it can't be SET, but can be SELECTed
        \begin{lstlisting}[ language=SQL
                          , deletekeywords={IDENTITY}
                          , deletekeywords={[2]INT}
                          , morekeywords={clustered}
                          , framesep=8pt
                          , xleftmargin=40pt
                          , framexleftmargin=40pt
                          , frame=tb
                          , framerule=0pt 
                          ]
CREATE TABLE <tname> ( <col1>
                     , <col2>
                     ...
                     , <colN> AS <col1> * <col2> PERSISTED
                     ) -- e.g.
                     -- PERSISTED is optional(it is stored rather than calculated on the fly)

ALTER TABLE <tname>
ADD <colname> AS (<col1> * <col2>) -- e.g.
        \end{lstlisting}
    \end{section}
    \begin{section}{partitions}
        \begin{lstlisting}[ language=SQL
                         , deletekeywords={IDENTITY}
                         , deletekeywords={[2]INT}
                         , morekeywords={clustered}
                         , framesep=8pt
                         , xleftmargin=40pt
                         , framexleftmargin=40pt
                         , frame=tb
                         , framerule=0pt 
                         ]
CREATE PARTITION FUNCTION <pfname> (<input param type>)
AS RANGE <LEFT | RIGHT> FOR VALUES(<boundary vals>)
-- LEFT | RIGHT is optional    
-- boundary vals are optional

CREATE PARTITION SCHEME <psname>
AS PARTITION <pfname> TO (<filegroups>)
        \end{lstlisting}
    \end{section}
    \begin{section}{geography and geometry types}
        \url{https://www.sqlshack.com/spatial-data-types-in-sql-
        server/}
    \end{section}
\end{document}